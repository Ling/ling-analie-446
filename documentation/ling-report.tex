\documentclass[12pt]{article}

\usepackage{setspace}

%\usepackage{pxfonts}
\usepackage[usenames,dvipsnames]{color}

\author{Ling Lin}
\title{Individual Experience Report}

\begin{document}
\maketitle

\section{What I learned in the project.}
During the course of the project I learned many things about c++, which I did
not know from the experience doing the assignments and reading the slides.


I learned much about project management.  I took this course concurrently with
the \emph{COMP354 - Software Engineering I} class.  I applied some concepts I
learned in that class.  I got to practice things like UML while discussing
points with my teammate, and elaborating our design.  The use was a little bit
frivolous, as we only had two classes and they didn't interact that much; most
of the functionnality ended up in the \texttt{Board} class.

I also learned the use of the gcc compiler more closely.  I learned about
seperating the building of the project into compiling the different
\emph{compilation units}, i.e. \textit{file}.cpp into the object files
\textit{file}.o or \textit{file}.obj using gcc's compiling facilities.  The
second step being to gather all the object files into an executable using gcc's
linking functionnality.

Finally, I learned some basic typesetting software, \LaTeXe, from a few
students in the 446 class, which I used to typeset this document as well as the
proposal.  I think it makes professional looking documents and will definitely
try to use it in preference to Microsoft's \emph{Word}. I also wish to learn
how to use it to create math formul\ae.

\section{What I wanted from the project.}
I had planned to use some type of version control for the development of the
project.  However it was too hard to understand and learn while trying to think
about the c++ I had to write for the team.  Unfortunately, my teammate and I
didn't really use the system.  We ended up exchanging zip files over email and
posting them on a google group.

Also, I really hoped I would be able to design and implement an algorithm to
solve a sudoku puzzle.  Again, I was not able to meet this expectation: our
current implementation only works sometimes, and doesn't solve the whole board.

\section{Thoughts About C++.}
\subsection{Likes}
I feel that c++ takes less code to do the same thing as it would in java.  It feels much simpler than java.  I like that starting out a new class or program involves much less \emph{support syntax}. C++ is cleaner such as:

\begin{verbatim}
class A{
  void b(){cout<<"Hello, world!<<endl
};

int main(){
  A a;
  a.b();
  return 0;
}
\end{verbatim}

against Java's:

\begin{verbatim}
public class A{
  public static void main(String[] ...
  ...
  void b(){...}
}
\end{verbatim}

The final thing I like is the seperation between the definition of a class in its header, and the definition of the methods in a \emph{cpp} file.

\subsection{Dislikes}
While I do like do be able to split the complete definition of a clas into
interface and implementation as we learned in class, I do find C++'s syntax
wierd.  Having learned Java as my first programming langue, I am more used to
its syntax.  I find it wierd that when you put stuff in the \emph{cpp} file you
have to add class:: in front of all the methods when you write their code.

I also find the \textit{<<} output syntax ugly and harder to read.  I prefer using + between strings.

\section{Thoughts About COMP 446.}

On a general note I am glad of choosing this class.  Peter Grogono, was the
best professor I've had at Concordia, the others I had this semester being very
bad. I appreciated the very quick feedback he gave on all class related things.

I enjoyed the chance to take this class.

\subsection{Liked}
I really enjoyed the class notes.  I found them easy to understand, and I
learned most of the material by reading them.  It was useful to review when
trying the assignments and I couldn't remember how to write something.
notes, well written, useful.

I liked the participation in the class.  Many people asked questions and helped
make some things that were less clear.

\subsection{Didn't like}

There are a few problem with the class.  I did not like the environment for the
presentation sometimes.  The combination of a dark room, with slide, and a
gentle voice lecturing made it easy to my mind to wander and lose track of the
class.

My other complaint was a few of the other students.  They kept asking questions
I didn't really understand, and I feel they were not really related to the
presentation.  I feel they already know c++ and asked questions that were not
on the same level as everyone else.
\end{document}
