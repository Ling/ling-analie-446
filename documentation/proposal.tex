\documentclass[12pt,letterpaper]{article}

\usepackage{pxfonts}

\title{Basic Sudoku}
\author{Analie-Jade Carri\`ere \and Ling Lin}
\begin{document}

\maketitle

\section{Introduction}

Sudoku is a number puzzle made up of a 9X9 grid subdivided into 9 3X3
boxes. The solved puzzle must contain numbers from 1 to 9 in each row,
each column and each 3X3 box which cannot be repeated. Usually, a
partially completed grid is presented to the user.

\section{Goals}

Our primary goal is to build a basic Sudoku game with two use cases.

\begin{itemize}
\item
The first case would be to generate a blank puzzle where the user can
input his/her own numbers. At this point, the user would be able to
choose one of two options. The first option would be to have the user
solve the puzzle. The second option would be to have the puzzle solve
itself.

The puzzle can be read from a file or directly input by the user. The
input can come from different sources such as newspapers or the
internet. Therefore, the user will be able to transcribe the newspaper
grid into either a file or directly into the grid. The input numbers
will then be locked into the grid and displayed in red. When the user
is ready to play, each of his/her moves will be validated by our
program.

\item
The second case would be to generate a puzzle displaying randomly
generated numbers. The user would then solve the puzzle.
\end{itemize}


\section{C++ Features}

The C++ features we will be using are STL, loops, iterators, stacks
and vectors.

 
\section{Incremental Deliverables}

\subsection{Basic version}

\begin{enumerate}
\item Display a blank grid using Ncurses.

\item Allow the user to build the initial grid by inputting numbers either
  by entering them directly into the grid or by importing them through
  a file.

\item The original numbers entered by the user will be locked and
  displayed in red.

\item Subsequently, each user input will be validated according to the
  current grid.

\item Write an algorithm to solve the puzzle.
\end{enumerate}

\subsection{Advanced version}

\begin{enumerate}

\item Automatically generate an initial puzzle with a unique solution, and
  let the user play.

\item Solve the puzzle and display the solution onto the screen or save it
  into a file.

\item Implement degrees of difficulty.

\end{enumerate}

\end{document}